\chapter{Introduction}

65\% of the worlds energy is produced using fossil fuels []. Those fuels have to main problems: they are not renewable, meaning that the resources might be depleted one day, and they have a negative CO2-balance [], resulting in a green house effect that changes the climate of the world[].

Due to this problems, alternatives are needed. Renewable production of energy using wind or solar combined with energy storage in batteries is a promising method already in use, but it can not replace fossil fuels in all situations. Generating biofuels from biomass allows for a direct replacement of fossil fuels, keeping all their benefits while still being renewable and C02 neutral. However, biomass production for fuel is not to compete with food production [...] .

One method to generate biomass without competing with food production is to grow algae in photobioreactors.

This photobioreactor is used to grow algae that produce lipids from carbon dioxide via photosynthesis. These lipids can be processed to biofuel and other oil-derivatives, replacing crude oil as precursor. In order to be economical viable, the reactor has to feature minimal investment and operating cost.

This means open bioreactors that would be located in regions where conventional farming isn't possible. By using saltwater there would also not bee competition for freshwater [?].

To better develop, compare and optimize different reactor concepts, a computational fluid dynamics model is being developed. In order to validate this model, and to generate data to feed into it, the real flow conditions in an actual reactor have to be studied.
The scope of this work is the development and test of a sensor system to make that possible. 