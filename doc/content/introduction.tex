\chapter{Introduction}

This photobioreactor is used to grow algae that produce lipids from carbon dioxide via photosynthesis. These lipids can be processed to biofuel and other oil-derivatives, replacing crude oil as precursor. In order to be economical viable, the reactor has to feature minimal investment and operating cost.

To better develop, compare and optimize different reactor concepts, a computational fluid dynamics model is being developed. In order to validate this model, and to generate data to feed into it, the real flow conditions in an actual reactor have to be studied.

The scope of this work is the development and test of a sensor system to make that possible. The method chosen beforehand was to measure the fluids electrical conductivity, which can be changed easily by adding water with differing salt concentrations. Commercially available conductivity meters are built to measure with high accuracy in order to obtain information about a liquids absolute salinity and relatively expensive. Our use case however does not need to create high accuracy absolute measurements, but measure a relative change allowing to distinguish two different liquids by their salinity. However, this needs to happen very fast and at a lot of different points in the stream. The more positions measured, the more complete the picture of the flow becomes. Therefore, the cost per sensor has to be low, to not put a restraint on the total number of points that can be measured.

The actual flow analysis is not part of this work, but rather the creation of a tool to make it possible. As such, the system needs to be designed to be used by others, not the creator himself. This thesis therefore describes product development rather than a scientific study.

The method to explore the flow conditions in the bioreactor used in this project is to measure the conductivity of the flowing water on multiple points with a high frequency. The conductivity is then changed by adding saltwater to the streaming freshwater, or by replacing the freshwater feed with a saltwater feed. The sensors then measure the increase in conductivity, signaling the arrival of the saltwater at certain positions. By mapping out the positions and the conductivity over time, an image of the flow can be generated. To get a usable image of the flow, the system has to meet certain requirements.

The goal of this project is the development and test of a low-cost electrical conductivity meter for liquids to be used as an aid to measure and analyze the flow in a photobioreactor.