\chapter{Appendix} 

\section{Instruction Manual} \label{aman}

\subsection*{Setting up the Hardware}

\begin{figure}[H]
	\begin{center}
		\tikzset{external/export next=false}
		\begin{tikzpicture}
			\node[,inner sep=0] at (0,0) {\includegraphics[width=0.5\textwidth]{images/cb.jpg}};
			\node[inner sep=0] at (0,-3) {\includegraphics[width=\textwidth]{images/fpcbp.jpg}};
			
			\draw[red,ultra thick,rounded corners] (-4,-0.55) rectangle (-3.25,0.55);
			
			%\draw[red!50!yellow,ultra thick,rounded corners] (-1,-0.35) rectangle (-0.25,0.35);
			%\draw[red!50!yellow,ultra thick,rounded corners] (0.55,-0.35) rectangle (1.3,0.35);
			
			\draw[red!25!yellow,ultra thick,rounded corners] (-1.45,-1) rectangle (-0.3,-0.6);
			\draw[red!25!yellow,ultra thick,rounded corners] (0.3,-1.05) rectangle (1.3,-0.65);
			\draw[red!25!yellow,ultra thick,rounded corners] (-1.54,1.05) rectangle (-0.44,0.65);
			\draw[red!25!yellow,ultra thick,rounded corners] (0.15,1.05) rectangle (1.15,0.65);
			
			%\draw[blue,ultra thick,rounded corners] (1.8,-1.2) rectangle (4,1.2);
			\draw[blue!75!white,ultra thick,rounded corners] (-8,-4) rectangle (8,-2);
			%\draw[blue!50!white,ultra thick,rounded corners] (-2.75,-3.5) rectangle (-1.65,-2.5);
		\end{tikzpicture}
		\caption[Important parts of the system.]{Important parts of the system:\\USB connection (\drawline[red,ultra thick])\\sensor strip (\drawline[blue!75!white,ultra thick])\\connectors (\drawline[red!25!yellow,ultra thick])}
		\label{fig:isysa}
	\end{center}
\end{figure}

\begin{itemize}
	\item[1] Use electrical tape to mount the sensor strips at desired points of measurement. Make sure not to cover the electrodes (golden parts).
	\item[2] Connect the cables of the sensor strip to the carrier board as shown in the picture:
	\begin{figure}[H]
	\begin{center}
		\includegraphics[width=0.5\textwidth]{images/conn.jpg}
		\caption{The connectors for the sensor strips.}
		\label{fig:iconna}
	\end{center}
\end{figure}
	\item[3] Connect the carrier board to the host PC with the USB cable. Note the orientation in the picture:
		\begin{figure}[H]
	\begin{center}
		\includegraphics[width=0.5\textwidth]{images/usb.jpg}
		\caption{The correct USB orientation.}
		\label{fig:iusba}
	\end{center}
\end{figure}
	Wrong orientation will not cause any damage, but the system will not work.
\end{itemize}

\subsection*{Capturing data using the OpenSalinityGUI}

\begin{itemize}
	\item[0] Follow the three steps of "Setting up the Hardware".
	\item[1] Start the OpenSalinity GUI by clicking on the icon.
	\item[2] Click "Save" on the top right to chose a file to log the data to. If you want to use the default filename, just click "Open" in the file menu.
	\item[3] Start the measurement by clicking "Start".
	\item[4] Stop the measurement by clicking "Stop".
	\item[5] Repeat from step 1 for all subsequent measurements. If no new file is chosen before starting a new measurement, the data is added at the end of the previous file.
\end{itemize}

\subsection*{Capturing data using command line tools}

\begin{itemize}
	\item[0] Follow the three steps of "Setting up the Hardware".
	\item[1] Open a terminal.
	\item[2] Change the working directory to the directory liveplot.sh is in:
	\begin{lstlisting}
		cd ~/path/to/directory
	\end{lstlisting}
	\item[3] Execute "liveplot.sh" with a file name as argument:
	\begin{lstlisting}
		./liveplot.sh log.csv
	\end{lstlisting}
	\item[4] To end data capturing, click into the terminal and press "Ctrl+C".
	\item[5] Repeat from step 1 for all subsequent measurements. If the file name is not changed, the existing file will be overwritten.
\end{itemize}

\newpage

\section{Algorithms}

\subsection*{Rolling Maximum} \label{rollmax}

The rolling maximum replaces all values with the local maxima within a slice of data. Listing \ref{lst:rm} shows a working implementation of the algorithm in the programming language Python.

\begin{lstlisting}[caption={Python implementation the rolling maximum algorithm.},label={lst:rm}]
import numpy as np

def roll_max(vector):

    for i in range(0, len(vector)):
        try:
            # replace each value with the maximum of the next 100 values
            vector[i] = max(vector[i+1:i+100])
        except:
            pass

    # shift vector 100 steps to the right
    vector = np.insert(vector, 1, [0]*100)[:-100]

    return vector
    
\end{lstlisting}

\subsection*{Signal Calibration} \label{sc}

The signal calibration scales all signals to a common reference. This common reference is the signal of the first sensor. Listing \ref{lst:sc} shows a Python implementation of the method.

\begin{lstlisting}[caption={Python implementation the signal calibration.},label={lst:sc}]
import numpy as np

sensor_data = [sensor_1, sensor_2, ...]		# array of all sensor data
start = 100		# start and
end = 1000		# end of the slice of undisturbed data

def sig_cal(sensor_data, start, end):

    means = []		# array to hold the means calculated from the slices

    for vector in sensor_data:
        mean = np.mean(vector[start:end])
        means.append(mean)	
        vector = vector * (means[0]/mean) 	# scale by ratio of mean of 
        																			#first sensor to current
        																			#sensor

    return sensor_data
    
\end{lstlisting}