%%%%%%%%%%%%%%%%%%%%%%%%%%%%%%%%%%%%%%%%%%%%%%%%%%%%%
%           Math hacks to simplify output           %
%                                                   %
%  Careful: Some hacks will ensure math mode        %
%      automatically, and thus can fail normal      %
%      expectations!                                %
%  Require package amsmath                          %
%%%%%%%%%%%%%%%%%%%%%%%%%%%%%%%%%%%%%%%%%%%%%%%%%%%%%
%% \bunderline
%%    An improved version of the underline command
%%
%%    Has one optional argument which defines how much
%%    the underline line is shortened. 
%%
%%    Example \underline[4]{<text>}
%%
%%    Source:
%%      http://tex.stackexchange.com/questions/49324/decrease-length-of-underline-in-math
\newcommand{\bunderline}[2][4]{\underline{#2\mkern-#1mu}\mkern#1mu}

%% \boverline
%%    An improved version of the overline command
%%
%%    Has one optional argument which defines how much
%%    the overline line is shortened
%%
%%    Source:
%%      http://tex.stackexchange.com/questions/22100/the-bar-and-overline-commands/22134#22134
\newcommand{\boverline}[2][1.5]{\mkern #1mu\overline{\mkern-#1mu#2\mkern-#1mu}\mkern #1mu}

%% \vect(or)
%%   Displays a variable as vector, can also be used
%%   in text environment. Three alternatives exist:
%%     - arrow vector notation
%%     - bold
%%     - underline
%%   Uncomment only the specific line to change format
%%   
%%   Careful: This command ensures math mode!
%%
%\newcommand{\vect}[1]{\ensuremath{\vec{#1}}}          % arrow
\newcommand{\vect}[1]{\ensuremath{\boldsymbol{#1}}}    % bold
%\newcommand{\vect}[1]{\ensuremath{\bunderline{#1}}}    % underline

%% \tensor
%%   Displays a variable as a tensor
%%   
%%   Careful: This command ensures math mode!
%%
%\newcommand{\tensor}[1]{\ensuremath{\boldsymbol{#1}}}            % bold
\newcommand{\tensor}[1]{\ensuremath{\bunderline{\boldsymbol{#1}}}} % bold underlined

%% \div, \Div
%%   Displays the mathematical operator of divergence.
%%   Faciliates brackets in capital version.
%%   Optional arguments
%%    #1 Type of parentheses
%%      Has no influence in \div version
%%      [square || <else>=round ]
%%
%%   Example: \Div[square]{\rho \vec{u}}
%%            \div{\vec{u}}
%%
%%   Careful: Overwrites the original div command!
%%
\renewcommand{\div}[2][1=round]{\nabla \cdot #2}
\newcommandx{\Div}[2][1=round]{%
  \ifthenelse{\equal{#1}{square}}{%
    % use square brackets
    \nabla \cdot \left[ #2 \right]
  }{%
    % use round braces
    \nabla \cdot \left( #2 \right)
  }
}

%% \grad, \Grad
%%    Displays the mathimatical operator of a gradient
%%    Faciliates brackets in captial version
\newcommand{\grad}[2][1=irrelevant]{\nabla #2}
\newcommandx{\Grad}[2][1=round]{%
  \ifthenelse{\equal{#1}{square}}{%
    % use square brackets
    \nabla \left[ #2 \right]
  }{%
    % use round braces
    \nabla \left( #2 \right)
  }
}

%% \mt
%%   Shortcut to insert text in math mode that is
%%   printed italic to mimic math font, but more 
%%   dense than normal math output.
\newcommand{\mt}[1]{\ensuremath{\text{\textit{#1}}}}

%% \parder, \Parder
%%    Displays a partial derivative
%%    Arguments:
%%      #2 Numerator
%%      #3 Demunerator
%%    Optional arguments:
%%      #1 Type of parantheses. Has no influence
%%        lower case version \parder
%%        'square' version will move the numerator
%%        after the fraction
%%        [square || <else>=round ]
%%
%%    Example: \parder{\vec{u}}{t}
%%             \Parder{\rho \vec{u}}{t}
%%             \Parder[square]{\rho \left( e + \frac{\vec{v}^2}{2}\right)}{t}
%%
\newcommand{\parder}[3][1=irrelevant]{\frac{\partial #2}{\partial #3}}
\newcommand{\Parder}[3][1=square]{
  \ifthenelse{\equal{#1}{square}}{
    \frac{\partial}{\partial #3} \left[ #2 \right]
  }{
    \frac{\partial \left(#2\right)}{\partial #3}
  }
}

%% \der, \Der
%%    Simple derivative
%%    Same options as \parder, \Parder
%%    but uses 'd' as derivative term instead of '\partial' (=delta)
\newcommand{\der}[3][1=irrelevant]{\frac{\mathrm{d}#1}{\mathrm{d}#2}}
\newcommand{\Der}[3][1=square]{
  \ifthenelse{\equal{#1}{square}}{
    \frac{\mathrm{d}}{\mathrm{d} #3} \left[ #2 \right]
  }{
    \frac{\mathrm{d} \left(#2\right)}{\mathrm{d} #3}
  }
}

%% \e
%%    Shortcut for printing 10^x coefficients
%%
%%    Example:  5\e{-3} -> 5*10^(-3)
%%
\newcommand{\e}[1]{\ensuremath{\cdot 10^{#1}}}

%% \exp
%%    Shortcut for the e^x notations
%%
%%  Example:    \exp{3} -> e^3
%%
\renewcommand{\exp}[1]{\mathrm{e}^{#1}}

%% \OD
%%    Shortcut for printing an OD
\newcommand{\OD}[2][1=750]{%
    \ensuremath{%
        % if #2 is empty just print OD_#1
        %   otherwise add the value
        \ifthenelse{\equal{#2}{}}{%
            \mathrm{OD}_{#1}%
        }{%
            \mathrm{OD}_{#1} = #2%
        }%
    }%
}

\DeclareMathOperator{\Var}{Var}

%% \transpose, \Transpose
%%    Shortcut to mark a matrix/tensor as transposed
%%    Uses the below defined \transposeSymbol
%%
%%    The uppercase variant checks for an optional first 
%%    parameter, which can either be "round" or "square"
%%    and will print parentheses or brackets around the 
%%    text.
%%
%%  Example:    \transpose{\tensor{\tau}}
%%              \Transpose[square]{\tensor{\tau}}
%%
\def\transposeSymbol{\intercal}
\newcommand{\transpose}[2][1=irrelevant]{\ensuremath{#2^\transposeSymbol}}
\newcommandx{\Transpose}[2][1=round]{%
  \ifthenelse{\equal{#1}{square}}{
    \ensuremath{\left[#2\right]^\transposeSymbol}
  }{%
    \ensuremath{\left(#2\right)^\transposeSymbol}
  }
}

\newcommand{\registered}{\textsuperscript{\textregistered}}

% the starred version of \ref will not create a link,
% but this fails anyway for all externalised images
\newcommand{\plotref}[1]{\tikzexternaldisable\ref*{#1}\tikzexternalenable}

\newcommand{\inlinetikz}[2][baseline=0]{%
  \tikzset{external/export next=false}%
  \tikz[#1]{#2}%
}

\newcommand{\imagelegend}[2][1=style]{%
  \ifthenelse{\equal{#1}{line}}{
    % #1 = line
    \inlinetikz{\draw [#2] (0, .25\baselineskip) -- (0.6, .25\baselineskip);}%
  }{
    % #1 = style
    \inlinetikz{\node [minimum size=0.5\baselineskip, #2, inner sep=0.1] at (0, 0.25\baselineskip) {};}%
  }
}

%% new the command for externalised tikz filenames
\newcommandx{\tikzsetnextfilenamecoloured}[2][1=colour]{%
  \ifthenelse{\equal{#1}{colour}}{
    % add a suffix tu determine wheter the graphic is colored or black and white
    \tikzsetnextfilename{#2\tikzExternalSuffix}
  }{
    \tikzsetnextfilename{#2}
  }
}

\pgfplotsset{ignore legend/.style={every axis legend/.code={
    \renewcommand\addlegendentry[2][]{} 
    \renewcommand\addlegendentryexpanded[2][]{}
    }}}

% \tablefootnote
%   This command can be used to manually create footnotes in a table
%   where it defines required extra spaces.
%   ! No automation at all
%
%   Use this command the following way:
%       Add "\textsuperscript{(2)}" in the table to reference footnote (2)
%           of the table below
%       Put the following code below the table to explain the footnotes 
%           you added
%
%           \tablefootnote{
%               \raggedright
%               \begin{tabular}{ll}
%               \multirow{3}{0.5cm}{} & \textsuperscript{(1)}Footnotetext 1 \tabularnewline
%                                     & \textsuperscript{(2)}Footnotetext 2 \tabularnewline
%                                     & \textsuperscript{(3)}Footnotetext 3
%               \end{tabular}
%           }
%       Adjust the number of multiple rows to the number of footnotes you 
%           have inserted. Furthermore you can manually adjust the space 
%           the first column takes up to center the footnotes
\newcommand{\tablefootnote}[1]{
    \vskip+\abovecaptionskip 
    \vskip-\belowcaptionskip
    \caption*{#1}
    \vskip-\abovecaptionskip
    \vskip+\belowcaptionskip
  }
  
\newcommand{\pictureheadnote}[1]{
    \vskip-\abovecaptionskip 
    \vskip+\belowcaptionskip
    \caption*{#1}
    %\vskip+\abovecaptionskip
    \vskip-\belowcaptionskip
  }
  
\let\originalparagraph\paragraph
\renewcommand{\paragraph}[2][.]{\originalparagraph{#2#1}}

% style to only print last n rows
\pgfkeys{
    /pgfplots/table/print last/.style={
        row predicate/.code={
            % Calculate where to start printing, use `truncatemacro` to get an integer without .0
            \pgfmathtruncatemacro\firstprintedrownumber{\pgfplotstablerows-#1} 
            \ifnum##1<\firstprintedrownumber\relax
                \pgfplotstableuserowfalse
            \fi
        }
    },
    /pgfplots/table/print last/.default=1
}

\pgfplotsset{
    simulation-plot/.style = {
        each nth point=10, filter discard warning=false,
        mark=none,
        x=Time,
        thick,
    },
    measurements-plot/.style = {
        mark=*, only marks, mark size=2pt, mark options={},
        x=Time,
        forget plot,
        error bars/y dir=both, error bars/y fixed relative=0.05,
    },
}

% pgfplots / pgfplotstable styles to filter tabledata,
% both copied from
% tex.stackexchange.com/questions/98003/filter-rows-from-a-table

\makeatletter
\pgfplotstableset{
    discard if not/.style 2 args={
        row predicate/.code={
            \def\pgfplotstable@loc@TMPd{\pgfplotstablegetelem{##1}{#1}\of}
            \expandafter\pgfplotstable@loc@TMPd\pgfplotstablename
            \edef\tempa{\pgfplotsretval}
            \edef\tempb{#2}
            \ifx\tempa\tempb
            \else
                \pgfplotstableuserowfalse
            \fi
        }
    }
}
\makeatother
    
\pgfplotsset{
    discard if not/.style 2 args={
    	x filter/.code={
    	    \edef\tempa{\thisrow{#1}}
    	    \edef\tempb{#2}
    	    \ifx\tempa\tempb
    	    \else
    	    \def\pgfmathresult{inf}
    	    \fi
    	}
    }
}

\usepackage[novbox]{pdfsync}

% arguments:
%   1: [order] (l, r, p, p!)
%   2: text
%   3: [width] ... of textbox
%   4: [spacing] ... between boxes
%   5: picture
\newlength{\picboxwidth}
\newcommandx{\besidespic}[5][1=p, 3=0.65\textwidth, 4=0.05\textwidth]{%
  {
  % need to put this in a local environment, otherwise values will be assigned globally
  \setlength{\picboxwidth}{\textwidth}
  \addtolength{\picboxwidth}{-#3}
  \addtolength{\picboxwidth}{-#4}
  \captionsetup{format=plain}
  
  % evaluate special cases position=p or p!
  %   set \pic to contain picture position
  \ifthenelse{\equal{#1}{p}}{
    % option p
    % notcomascript-documents
    %\ifthenelse{\isodd{\thepage}}{
    % Koma Script based documents
    \ifthispageodd{
      % odd
      \def\pic{r}
    }{
      % even
      \def\pic{l}
    }
  }{
    \ifthenelse{\equal{#1}{p!}}{
      % option p!
      % notcomascript-documents
      %\ifthenelse{\isodd{\thepage}}{
      % Koma Script based documents
      \ifthispageodd{
        % odd
        \def\pic{l}
      }{
        % even
        \def\pic{r}
      }
    }{
      \def\pic{#1}
    }
  }
  
  %% Output text and picture
  \ifthenelse{\equal{\pic}{l}}{
    \begin{minipage}[c]{\picboxwidth}
      #5
    \end{minipage}\hspace{#4}
    \begin{minipage}[c]{#3}
      #2
    \end{minipage}
  }{
    \begin{minipage}[c]{#3}
      #2
    \end{minipage}\hspace{#4}
    \begin{minipage}[c]{\picboxwidth}
      #5
    \end{minipage}
  }
  
  % end of seperate environment
  }
}

%%%%%%%%%%%%%%%%%%%%%%%%%%%%%%%%%%%%%%%%%%%%%%%%%%%%%
% Bibliography use brackets [] instead of braces () %
%%%%%%%%%%%%%%%%%%%%%%%%%%%%%%%%%%%%%%%%%%%%%%%%%%%%%
\makeatletter

\newrobustcmd*{\parentexttrack}[1]{%
  \begingroup
  \blx@blxinit
  \blx@setsfcodes
  \blx@bibopenparen#1\blx@bibcloseparen
  \endgroup}

\AtEveryCite{%
  \let\parentext=\parentexttrack%
  \let\bibopenparen=\bibopenbracket%
  \let\bibcloseparen=\bibclosebracket}

\makeatother

\DeclareCiteCommand{\citetitlelink}
  {\boolfalse{citetracker}%
   \boolfalse{pagetracker}%
   \usebibmacro{prenote}}
  {\ifciteindex
     {\indexfield{indextitle}}
     {}%
   \printtext[bibhyperref]{\printfield[citetitle]{labeltitle}}}
  {\multicitedelim}
  {\usebibmacro{postnote}}

%%%%%%%%%%%%%%%%%%%%%%%%%%%%%%%%%%%%%%%%%%%%%%%%%%%%%
% Shortcuts for company names with registered char  %
%%%%%%%%%%%%%%%%%%%%%%%%%%%%%%%%%%%%%%%%%%%%%%%%%%%%%
% print ANSYS Fluent name with registered trademark symbol
\newcommand{\fluent}{ANSYS\textsuperscript{\textregistered} Fluent\textsuperscript{\textregistered}}
% print ANSYS ICEM name with registered trademark
\newcommand{\icem}{ANSYS\textsuperscript{\textregistered} ICEM\textsuperscript{\textregistered}}
% print OpenFOAM with registered trademark symbol
\newcommand{\openfoam}{OpenFOAM\textsuperscript{\textregistered}}
% print ansys without a specific product
\newcommand{\ansys}{ANSYS\textsuperscript{\textregistered}}
% print Matlab with registered trademark symbol
\newcommand{\matlab}{MATLAB\textsuperscript{\textregistered}}

%%%%%%%%%%%%%%%%%%%%%%%%%%%%%%%%%%%%%%%%%%%%%%%%%%%%%
%    Have abbreviation shortcuts to avoid end of    %
%             sentence spacing problems             %
%%%%%%%%%%%%%%%%%%%%%%%%%%%%%%%%%%%%%%%%%%%%%%%%%%%%%
\makeatletter
\newcommand{\etc}{etc\@ifnextchar.{}{.\@}}      % etc.
\newcommand{\eg}{e.g\@ifnextchar.{}{.\@}}       % e.g.
\newcommand{\ie}{i.e\@ifnextchar.{}{.\@}}       % i.e.
\makeatother

%%%%%%%%%%%%%%%%%%%%%%%%%%%%%%%%%%%%%%%%%%%%%%%%%%%%%
%        Declare special unicode characters         %
%%%%%%%%%%%%%%%%%%%%%%%%%%%%%%%%%%%%%%%%%%%%%%%%%%%%%
%\DeclareUnicodeCharacter{010C}{\v{C}}  % Č

\widowpenalty10000
\clubpenalty10000