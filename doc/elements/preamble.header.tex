%% Sprachanpassungen
%\usepackage[ngerman]{babel}
\usepackage[british]{babel}
\usepackage[T1]{fontenc}
\usepackage[utf8]{inputenc}
\usepackage[gen]{eurosym}

%% Textausgabe/-darstellung
\usepackage{lmodern} 				% Type1-Schriftart f�r nicht-englische Texte
\usepackage[tight,nice]{units}		% Einheitenformatierung
\usepackage{multirow}				% Mehrzeilige Tabellenzellen
\usepackage[onehalfspacing]{setspace}	% 1.5-facher Zeilenabstand
\usepackage{booktabs}				% Tabellen
\usepackage{enumerate}
\usepackage[usenames,dvipsnames,svgnames,table]{xcolor}
\usepackage{longtable}
\usepackage{listings}
\usepackage[hyphens]{url}

\usepackage{fontspec}
\defaultfontfeatures{Mapping=tex-text,Scale=MatchLowercase}
\setmainfont{Source Serif Pro}
\setmonofont{Source Serif Pro}

\newcommand*\diff{\mathop{}\!\mathrm{d}}
\usepackage[pdf]{pstricks}
\usepackage{tabularx}
\usepackage[europeanresistors,americaninductors]{circuitikz}

%% Mathemodus
\usepackage{amsmath,amssymb}
\usepackage{icomma}

%% Packages f�r Grafiken & Abbildungen
% Importing of graphics into the pdf
\usepackage{graphicx}
% creating of graphics / plots / extended color mixing possibilities
\usepackage{tikz,pgfplots,xxcolor} 
% reading and scripting tables automatically
\usepackage{pgfplotstable}
\usepackage{float,subcaption} 		% Teilabbildungen in einer Abbildung
\usepackage{floatflt}			% Bilder/Tabellen von Text umflossen
\usepackage[section]{placeins} 	% Haelt floats davon ab ueber Sections zu springen.
\usepackage[
		font=footnotesize,
		format=hang,
		indention=0.0cm,
		justification=justified,
		labelfont=bf,
	]{caption}					% Darstellung von Titeln
%\captionsetup{singlelinecheck=off}
\captionsetup[subfigure]{font=footnotesize}

%% Seitengestaltung
\usepackage{fancyvrb}
\usepackage[automark,headsepline]{scrlayer-scrpage}
\usepackage[
	bookmarks=true,
	hypertexnames=false,
    hyperfootnotes=true
	]{hyperref}
\usepackage{bookmark}
\usepackage[nodayofweek]{datetime}


%% Sonstiges
\usepackage{remreset}			% Kein Reset von Countern
	\makeatletter
		\@removefromreset{footnote}{chapter}
	\makeatother
\usepackage{scrhack}

%% Bibliographie
\usepackage[german=quotes]{csquotes}
\usepackage[
	backend=bibtex,
	%style=authoryear,
	bibstyle=authoryear,
	dashed=false,
	mergedate=false,
	maxbibnames=99,
	maxcitenames=1,
	hyperref=true,
	doi=false,
	url=false,
	isbn=false,
	eprint=false
	]{biblatex}
\usepackage{array}

% Code logic
\usepackage{xargs}
\usepackage{ifthen}


