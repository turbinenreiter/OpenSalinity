\chapter{Conclusion}

The goal of the project was the development of a low-cost electrical conductivity meter for liquids to measure and analyze the flow in a photobioreactor. After a clear set of requirements regarding the spacial resolution, sensitivity, price and usability were derived from the initial goal, a summary of the theoretical background was provided. A market research identified possible solutions and surfaced a product called MinieC suitable for the task. Around this part the necessary electronics and software were developed to take measurements as well as capture and visualize the data. A suitable solution for the manufacturing of sensors made from electrode pairs was found with flexible printed circuit boards. The resulting device was tested and analyzed to review the compliance with the formulated requirements. The requirement for the measurement range was not met, but the later validation still succeeded, which means it was higher than necessary. The validation was done by conducting several experiments on the reactor and analyzing the resulting data. The behavior indicated by the measurements met the expectations and allowed to derive information about the flow conditions.

\chapter{Outlook}

Further development of the sensor system can be done to solve some of the weaknesses of the current system. Due to the nature of the MinieeC in combination with the need of a fast input respond time, the data signal is troubled by high levels of noise. While digital filtering of the captured information makes it usable, these methods always result in a loss of information. One possibility to mitigate this issue is by further increasing the sample rate and removing the diode from the MinieC that filters the negative half of the wave. Doing so would allow to measure the complete oscillating signal, instead of random samples of it. This would also enable to not only measure the amplitude of the signal but also its phase relative to the input, allowing for measuring not only the real part of the resistance, but also the imaginary part, resulting in a complete measuring of the impedance. In order to accomplish this, according to the Nyquist-Shannon sampling theorem, the sample rate would have to be at least two times the oscillation frequency of the input signal, resulting in a sample rate of \unit[3.2]{kHz}. The existing hardware is capable of that, but the software would need a complete rewrite. A second option to solve this problem is using a dedicated impedance sensor like the AD5933 from Analog Devices. This sensor integrates the same measurement principle the MinieC uses on a single chip, alongside additional logic that analyzes the signal's amplitude and phase to calculate the complex impedance value. A PCB incorporating this part in a fashion compatible with the pin layout of the already developed carrier board is possible, providing a simple update strategy.