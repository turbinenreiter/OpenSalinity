\tikzexternaldisable
  
% Uses colormaps, in this case import from
%	colormaps example
% Colormap changing from blue over
%	white to red for positive and negative
%	sensitivities
\pgfplotsset{
  /pgfplots/colormap={neg to pos}{
    rgb255(0pt)=(0, 0, 255);
    rgb255(500pt)=(255, 255, 255)
    rgb255(1000pt)=(255, 0, 0);
    }
}

% Colormap changing from white to red,
%	for positive sensitivities only
\pgfplotsset{
  /pgfplots/colormap={positive only bright}{
    rgb255(0pt)=(255, 255, 255);
    rgb255(1000pt)=(255, 0, 0);
    }
}

% Colormap changing from black to red,
%	for positive sensitivities only
\pgfplotsset{
  /pgfplots/colormap={positive only dark}{
    rgb255(0pt)=(0, 0, 0);
    rgb255(1000pt)=(255, 0, 0);
    }
}

% filenname of the input matrix and label descriptions
%   requires three files:
%	<name>.csv containts the matrix, including colormap data and values
%	<name>.xNames contains the labels for the x-axis and its numer of
%			rows defines the length of the x-axis
%	<name>.yNames contains the labels for the y-axis and the length
%			of it
\def\matrixname{data/pgfplots+sensitivity-matrix}

\begin{tikzpicture}
  % define backgroundcolor
  \colorlet{background-color}{black!0!white}

  % Read data
  \pgfplotstableread[col sep=semicolon]{\matrixname.xNames}\xnametable
  \pgfplotstablegetrowsof{\xnametable}
  \let\numOfCols\pgfplotsretval
  
  \pgfplotstableread[col sep=semicolon]{\matrixname.yNames}\ynametable
  \pgfplotstablegetrowsof{\ynametable}
  \let\numOfRows\pgfplotsretval
  
  \pgfplotstableread[col sep=semicolon]{\matrixname.csv}\matrixtable
  
  % get minimum and maximum value in matrix
  \pgfplotstablesort[sort key={value}, sort cmp={float <}]{\sortedmatrixtable}{\matrixtable}
  \pgfplotstablegetelem{0}{value}\of\sortedmatrixtable
  \let\minValue\pgfplotsretval
  
  \pgfplotstablegetrowsof{\sortedmatrixtable}
  \pgfmathsetmacro{\lastrow}{\pgfplotsretval-1}
  \pgfplotstablegetelem{\lastrow}{value}\of\sortedmatrixtable
  \let\maxValue\pgfplotsretval
  
  % Get actual borders of min and max Values
  \pgfmathsetmacro{\minValue}{-max(abs(\minValue),abs(\maxValue))}
  \pgfmathsetmacro{\maxValue}{max(abs(\minValue),abs(\maxValue))}
  
  % Must define range of colorvalues, regardless of what the csv file contains,
  %  thus can not be calculated
  \def\minColor{0}
  \def\maxColor{100}
  
  \pgfmathsetmacro{\diagHeight}{0.55*\numOfRows}
  \pgfmathsetmacro{\diagWidth}{0.55*\numOfCols}

	% And draw
	\begin{axis}[
      ymin=1, ymax=\numOfRows,
      xmin=1, xmax=\numOfCols,
      height={\diagHeight cm},
      width={\diagWidth cm},
      enlargelimits={abs=1},
      scale=1.2,
      unit vector ratio*=1 1 1,
      y dir=reverse,
      colorbar, 
      colormap name=neg to pos,
      colorbar style={
        yticklabel={
          \tiny
          \pgfmathparse{(\tick-\minColor)/(\maxColor-\minColor)*(\maxValue-\minValue)+\minValue}
          $\pgfmathprintnumber{\pgfmathresult}$
          },
        ylabel=Sensitivity Index,
        font=\footnotesize
        },
      mark options={draw=none},
      axis background/.style={fill=background-color},
      xtick={1, ..., \numOfCols},
      ytick={1, ..., \numOfRows},
      xticklabel={$\pgfplotstablegetelem{\ticknum}{xNames}\of\xnametable\pgfplotsretval$},
      yticklabel={$\pgfplotstablegetelem{\ticknum}{yNames}\of\ynametable\pgfplotsretval$},
      x tick label style={font=\tiny, rotate=50, anchor=north east, yshift=0.1cm, xshift=0.1cm},
      y tick label style={font=\tiny, rotate=40, anchor=east, xshift=0.1cm, yshift=0.1cm},
      xlabel=x-axis,
      ylabel=y-axis,
    ]
    
    \addplot+[
        scatter,
        scatter src=explicit,
        faceted color=blue,
        only marks, 
        scatter/use mapped color={draw=background-color, fill=mapped color},
        mark options={mark=square*, scale=2.7, draw opacity=1},
	  %% Descide which line to use
	  % Uses the value column in the sensitivity matrix for the colormap
	  %		and calculates a logarithmic matrix
      %] table [x=x, y=y, meta expr={ln(\thisrow{value})/ln(10)}, col sep=semicolon] 
	  
	  % Uses value column in a non-logarithmic visualisation
	  %] table [x=x, y=y, scatter src= explicit meta=value, col sep=semicolon] 
	  
	  % Uses the color-column for the colormap. Matrix has to be prepared
	  %		to contain the correct data
      ] table [x=x, y=y, scatter src=explicit, meta=color, col sep=semicolon, point meta min=\minColor, point meta max=\maxColor] 
        {\matrixtable};
  \end{axis}
\end{tikzpicture}
